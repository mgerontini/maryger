\section{Introduction}\label{sec:introduction}             % chapter 1
Word-of-mouth marketing has always been an important success factor for consumer oriented businesses: brand reputation is an important part of the value of a company. Today, with the advent of social media, the reputation of a brand, product or service can change much more rapidly than before, and the range of consumer sentiment and attitude is much larger than before \cite{pang2008opinion}.

The reputation of brands, products and services have historically been difficult to track. Today it can be tracked by what is written about them online, e.g., in micro-blogs such as Twitter. Brand reputation mining by monitoring social media sources for language that may affect reputation in a positive or negative way, can be a powerful tool for an organization's public relations and marketing departments.

To address the aforementioned opportunity, an online reputation management system was developed. The system essentially takes a keyword (the name of a company, for example) and categorizes people's opinions extracted from Twitter messages\footnote{So called "tweets".} as positive, negative or neutral. Moreover, it measures how strong people's attitudes are, and takes into account how effective and authoritative those people are on Twitter. The system utilizes techniques in machine learning, information retrieval and natural language processing, and the results are conveyed to the reader in a very simple and understandable manner.

Annotated tweets were used to test the precision of the sentiment analysis algorithm. However, sentiment is not an absolute measure: even human annotators will often disagree, achieving no more than 82\% agreement in an earlier study for example \cite{wilson2005recognizing}. An abundance of data alleviates this problem to some degree: if an algorithm agrees with a human in more than half the cases, the integrated result will reflect the true polarity of an entity, given a sufficient number of tweets.